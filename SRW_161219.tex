\documentclass[technicalreport]{ieicej}
%\documentclass[technicalreport,usejistfm]{ieicej}
\usepackage[dvipdfmx]{graphicx}
\usepackage{latexsym}
%\usepackage[fleqn]{amsmath}
\usepackage{amsmath}
\usepackage{amssymb}
\usepackage{multirow,eepic}
\usepackage{cite}
\usepackage{mediabb}
\usepackage{url}
\usepackage{comment}
\usepackage{epsfig}
\usepackage{subfig}
\setlength{\oddsidemargin}{-9mm}
\setlength{\evensidemargin}{-9mm}
\setlength{\topmargin}{-4mm}
%\renewcommand{\topfraction}{1.0}
%\renewcommand{\bottomfraction}{1.0}
%\renewcommand{\dbltopfraction}{1.0}
%\renewcommand{\textfraction}{0.01}
%\renewcommand{\floatpagefraction}{1.0}
%\renewcommand{\dblfloatpagefraction}{1.0}
\setcounter{topnumber}{5}
\setcounter{bottomnumber}{5}
\setcounter{totalnumber}{10}

\newcommand{\sij}{(i,j)}
\newcommand{\sijk}{(i,j,k)}
\newcommand{\mN}{{\mathcal N}}
\newcommand{\pijk}{p^{(i,j,k)}}
\newcommand{\rd}{r^{\sij}_{\rm d}}
\newcommand{\ru}{r^{\sij}_{\rm u}}
\newcommand{\rijk}{r^{(i,j,k)}}
\newcommand{\etau}{\eta_{\rm u}^{(j)}}
\newcommand{\mthc}{\mathcal C}
\def\coloneqq{\mathrel{\mathop:}=}

\jtitle{無線LANにおける全二重通信と上りOFDMAの併用による遅延時間削減}
%\jsubtitle{}
\etitle{Delay Reduction Using In-Band Full-Duplex system and Uplink OFDMA system for wireless LAN}
%\esubtitle{}
 \alignorder=4
 \breakauthorline{4}

\authorlist{%
 \authorentry[info16@imc.cce.i.kyoto-u.ac.jp]{飯田 直人}{Naoto Iida}{^1}
 \authorentry{西尾 理志}{Takayuki NISHIO}{^1}
 \authorentry{守倉 正博}{Masahiro MORIKURA}{^1}
 \authorentry{山本 高至}{Koji YAMAMOTO}{^1}
 \authorentry{鍋谷 寿久}{Toshihisa Nabetani}{^2}
 \authorentry{青木 亜秀}{Tsuguhide Aoki}{^2}
 }

\affiliate[^1]{京都大学大学院情報学研究科\hskip1zw
  〒606-8501 京都市左京区吉田本町}
 {Graduate School of Informatics, Kyoto University\hskip1em
  Yoshida-honmachi, Sakyo-ku, Kyoto,
  606-8501 Japan}
\affiliate[^2]{株式会社東芝 研究開発センター\hskip 1zw 〒212-8582 神奈川県川崎市幸区小向東芝町1}
  {Corporate Research \& Development Center, TOSHIBA Corporation\hskip1em
   1 Komukaitoshiba-cho, Saiwai-ku, Kawasaki-shi, Kanagawa 212-8582, Japan}

\begin{document}
\begin{jabstract}
	無線LANの大容量化に向けて,自己干渉除去技術により送信と受信を同一帯域で同時に行う全二重通信無線LANが有望である.
	特に,1台のAP(Access Point)とAPへの上り通信を行うSTA(Station),
	APからの下り通信を受信するSTAの3台によるUFD(user-multiplexing Unidirectional Full-Duplex)通信は自己干渉技術を持たないSTAにも適用可能であるという利点がある.
	このUFD通信では,APの受信信号にAPの送信信号が影響を与える自己干渉と,
	STAによる上り通信信号がAPによる下り通信の送信先STAにおいて下り通信信号と干渉するユーザ間干渉が存在する.
	これらの干渉により,同時に行う上下通信のそれぞれの伝送速度は半二重通信の場合のそれに比べて低くなってしまう.
	その結果,フレームの時間長が長くなり遅延時間が長くなってしまうという問題がある.
	本稿ではこの問題の解決のために,UFD通信とOFDMAを組み合わせる方式を提案する.
	具体的には,遅延時間を考慮した最適化問題を設計し,それを解くことで,確率的に通信方式の選択,通信に参加するSTAの選択を行う.
	更に,本手法の有効性を計算機シミュレーションにより評価する.
\end{jabstract}
\begin{jkeyword}
	全二重通信無線LAN,ユーザ間干渉,公平性,遅延時間, OFDMA
\end{jkeyword}
\begin{eabstract}
	Self-interference cancellation techniques enable in-band full-duplex wireless local area networks (WLANs), where transmit and receive simultaneously in the same frequency band.
	In particular, user-multiplexing unidirectional full-duplex (UFD) communication, where a full-duplex access point (AP) transmits a frame to a station (STA) and receives a frame from another STA, is seen as an enabler of the WLANs, since UFD is applicable for the conventional STAs without self-interference cancellation capability.
	However, the UFD communication has a large delay of STAs because low physical(PHY) rates are used due to the interferences.
	In this paper, we propose a scheme to mitigate delay.
	The scheme use the UFD communication and uplink OFDMA system, and increase transmission opportunities of STAs.
	Simulation results show that the scheme reduce the delay which is the same level with the half-duplex system.

\end{eabstract}
\begin{ekeyword}
	full-duplex wireless LAN, inter-user interference, fairness, delay, OFDMA
\end{ekeyword}

\titlepagebaselinestretch{0.95}

\maketitle

\section{はじめに}
	近年,無線LAN(Local Area Network)が急速に普及し,急増するトラヒックにより2.4\,GHz帯は逼迫しており,
	近い将来5\,GHz帯の逼迫も懸念されることから,無線LANのさらなる大容量化は急務である.
	大容量化を実現する方法の一つとして,送信と受信を同じ周波数帯で同時に行う全二重通信無線LANが有望である.
	全二重通信無線LANでは上り通信と下り通信を同一帯域で同時に行うため,システムスループットを向上させることができる.
	全二重通信無線LANにはAP(Access Point)の送信先STA(station)とAPへの上り通信を行うSTAが同じである双方向全二重(BFD: Bidirectional Full-Duplex)通信と,
	図\ref{fig:topology}\subref{fig:ufd}に示すような上り通信を行うSTAと下り通信を受信するSTAが異なる全二重(UFD: user-multiplexing Unidirectional Full-Duplex)通信がある.
	UFD通信は半二重通信にしか対応していないSTAにも適用可能であるという利点がある.
	このUFD通信を用いた無線LANでは二つの干渉が問題となる.
	一つは,送受信を行っているAPにおいて,送信信号が所望の受信信号に干渉を及ぼす自己干渉である.
	自己干渉を受けるAPにとって干渉波は自身の送信信号であり既知であるため,
	自己干渉除去技術を用いて自己干渉を最大110\,dB除去できることが示されている~\cite{fdmac, stanford1}.
	もう一つは,STA $j$の送信信号がもう一方のSTA $i$の受信信号に干渉を及ぼすユーザ間干渉である.
	ユーザ間干渉では,干渉を受けるSTA $i$にとって干渉波がSTA $j$の送信する未知の信号であるため,
	自己干渉のように除去することができない.
	\par
	このユーザ間干渉の影響を低減するためには,干渉の大きさを考慮して上り通信を行うSTAと下り通信を受信するSTAの適切な組み合わせを選び出すことや,
	送信電力制御を行うことが必要である.これ関し,システムスループットを最大化することを目的関数とした最適化問題を解くことでSTA選択し,その後電力制御を行う手法が提案されている~\cite{promac}.
	更に,筆者らはこの手法を拡張し,システムスループットだけでなくSTA間の公平性やQoSを改善する手法を提案している~\cite{promac_fair}.
	\par
	しかし,UFD通信では半二重通信を用いる場合に比べ,STAの遅延時間が増大してしまうという問題がある.
	これは,ユーザ間干渉の影響により,半二重通信の場合に比べ伝送速度が低下してしまうため,
	データフレームの時間長が長くなることが原因である.
	\par
	本稿では,遅延時間削減のために図\ref{fig:topology}\subref{fig:ufd_ofdma}のようにUFD通信と上りOFDMAを組み合わせる手法を提案する.
	上り通信をOFDMAによって多重化しSTAの送信機会を増加させることでSTAの遅延時間を削減する.
	具体的には,~\cite{promac_fair}における最適化問題,STA選択手法をOFDMAを適用できるように拡張する.
	更に,本手法の有効性を計算機シミュレーションにより評価する.
	\par
	本稿の構成は以下のとおりである.第2章で本稿で扱うシステムモデルについて述べ,
	第3章では従来方式の問題点について述べる.
	更に,第4章において提案方式について述べ,第5章では提案方式の有効性をシミュレーションによって評価する.
	最後に第6章でまとめとする.

	\begin{figure}[t]
		\centering
		\subfloat[UFD通信]{
			\epsfig{file=fig/model_relay.eps, scale=0.4}
			\label{fig:ufd}
		}
		\\
		\subfloat[UFD通信とOFDMA通信の組み合わせ]{
			\epsfig{file=fig/ofdma.eps, scale=0.4}
			\label{fig:ufd_ofdma}
		}
		\caption{各通信方式}
		\label{fig:topology}
	\end{figure}

\section{システムモデル}\label{seq:system}
	\begin{figure}[t]
		\centering
		\epsfig{file=fig/pos.eps, scale=0.6}
		\caption{システムモデル(中心に設置されたAPとランダムに配置されたSTA)}
		\label{fig:model}
	\end{figure}

	本稿で検討するシステムモデルを図\ref{fig:model}に示す.
	1台のAPが$L$\,m四方の領域の中心に設置され,その周りに$N$台のSTAがランダムに配置されているとする.
	それらSTAのインデックス集合を$\mN=\{1,2,...,N\}$とする.
	OFDMAによる多重化は二多重までとし,
	この$N$台のSTAの中から,図\ref{fig:topology}\subref{fig:ufd_ofdma}のようにAPからの下り通信を受信するSTA $i$と,
	APへの上り通信を行うSTA $j$,$k$を選び出す.
	このとき,STAの組み合わせを$\sijk$と表現し,$i,\ j,\ k \in \{0\}\cup \mN$であり,
	STAは自己干渉除去技術を持たずBFD通信はできないと仮定して$i\neq j$かつ$i\neq k$とする.
	また,$i$,$j$,$k$のすべてが0になることはないものとする.
	$i$,$j$,$k$のそれぞれが取る値によって以下の通信方式を示し,切り替え可能であるものとする.
	\begin{description}
		\item[\hspace{15pt}$i=0\cap ((jk=0 \cap j+k\neq0) \cup(j=k\neq0))$の場合]\mbox{}\\
			\hspace{30pt}上りの半二重通信
		\item[\hspace{15pt}$i\neq0 \cap j=k=0$の場合]\mbox{}\\
			\hspace{30pt}下りの半二重通信
		\item[\hspace{15pt}$i\neq0 \cap ((jk\neq0 \cap j+k\neq0) \cup (j=k\neq0))$の場合]\mbox{}\\
			\hspace{30pt}UFD通信
		\item[\hspace{15pt}$i=0 \cap jk\neq0 \cap j\neq k$の場合]\mbox{}\\
			\hspace{30pt}上りOFDMA
		\item[\hspace{15pt}$i\neq0 \cap j\neq0 \cap k\neq0$の場合]\mbox{}\\
			\hspace{30pt}OFDMAとUFD通信の組み合わせ
	\end{description}
	\par
	また,STAの組み合わせを決定する際に用いる推定スループットには,上り下り通信のそれぞれのSINR(Signal-to-Interference plus Noise power Ratio)から求めた以下のシャノン容量$C$を用いる.
	\begin{equation}
		C=B\log_2(1+{\rm SINR}) \label{eq:capacity}
	\end{equation}
	ただし,$B$は通信に用いる帯域幅である.

\section{従来方式の問題点とその原因}\label{sec:problem}
	従来のUFD通信を用いる方式では,半二重通信である場合に比べて遅延時間が増大してしまう.
	本節では遅延時間の増加という問題点とその原因について述べる.
	\subsection{遅延時間の増加}
		\begin{figure}[t]
			\centering
			\epsfig{file=graph/delay_sub.eps, scale=0.6}
			\caption{半二重通信と従来方式における送信待機時間}
			\label{fig:delay_sub}
		\end{figure}
		図\ref{fig:delay_sub}に半二重通信のみを用いる場合と半二重通信とUFD通信を併用する従来方式~\cite{promac_fair}の場合の
		STAの平均送信待機時間を示す.
		ただし,送信待機時間とは各STAにおいてあるデータフレームがバッファの先頭に到着してからの経過時間であり,
		$\alpha$は目的関数における送信待機時間の.影響を調整するパラメータである.
		送信待機時間が半二重通信の場合と比べて,従来方式は4倍以上の値となっていることがわかる.
	\subsection{伝送速度の低下}
		本節では送信待機時間が半二重通信の場合と比べて大きくなってしまう原因について述べる.
		APの送信電力を$P_{\rm AP}$,STA $j$の送信電力を$P_j$とし,
		AP-STA $i$間,AP-STA $j$間,STA $i$-$j$間のチャネル係数をそれぞれ$h_{{\rm AP},i}$,$h_{{\rm AP},j}$,$h_{i,j}$とし,
		雑音電力を$\sigma^2$とすると,半二重通信の場合における上下通信のSNRはそれぞれ,
		\begin{align}
			{\rm SNR_{\rm d}^{\rm h}} &= \cfrac{P_{\rm AP}|h_{{\rm AP},i}|^2}{\sigma^2} \label{eq:hd}\\
			{\rm SNR_{\rm u}^{\rm h}} &= \cfrac{P_j |h_{{\rm AP},j}|^2}{\sigma^2}\label{eq:hu}
		\end{align}
		となる.
		一方,UFD通信の場合の上下通信のSINRはそれぞれ,
		\begin{align}
			{\rm SINR_{\rm d}^{\rm f}} &= \cfrac{P_{\rm AP} |h_{{\rm AP},i}|^2}{\sigma^2+P_j'|h_{i,j}|^2}\label{eq:fd}\\
			{\rm SINR_{\rm u}^{\rm f}} &= \cfrac{P_j' |h_{{\rm AP},j}|^2}{\sigma^2+P_{\rm AP}|h_{\rm AP,AP}|^2}\label{eq:fu}
		\end{align}
		となる.
		ただし,$P_j'$は~\cite{promac}による送信電力制御を行った場合のSTA $j$の送信電力であり$P_j'\leq P_j$である.
		また,$h_{\rm AP,AP}$はAPの送信信号がAP自身によって受信されるまでの伝搬路のチャネル係数に,
		自己干渉除去技術による干渉除去の効果を加味した値である.
		\par
		式\eqref{eq:hd}と\eqref{eq:fd}の比較,式\eqref{eq:hu}と\eqref{eq:fu}の比較からわかるように,
		UFD通信における上下通信のSINRは,半二重通信の場合のSNRと比べて干渉と送信電力制御による送信電力低下の分だけ悪化する.
		これにより,UFD通信では,半二重通信の場合に比べて上下通信それぞれの伝送速度が低下してしまう.

\section{提案方式}\label{sec:propose}
	本節では,遅延時間の増大を防ぐため,上り通信をOFDMAを用いて多重化することを提案する.
	OFDMAを用いてSTAの送信機会を増加させることで,前節で述べた伝送速度の低下による遅延時間の増大を軽減する.
	STA決定手法は従来方式~\cite{promac_fair}における手法をOFDMAを適用できるように拡張する.
	また,計算時間削減のための手法についても検討する.
	\subsection{STAの組み合わせ決定手法}\label{sec:opt}
		%まず,STAの組み合わせの集合${\mathcal C}_{\rm half}$,${\mathcal C}_{\rm UFD}$,${\mathcal C}_{\rm OFDMA}$,${\mathcal C}_{\rm OFDMA-UFD}$,${\mathcal C}$を以下のように定義する.
		%\begin{align}
		%	&{\mathcal C}_{\rm half} \coloneqq \{\sijk : i=0\cap ((jk=0 \cap j+k\neq0) \cup(j=k\neq0)),\ \rijk >\epsilon\} \\
		%	&{\mathcal C}_{\rm UFD} \coloneqq \{\sijk : i,j\in{\mathcal N},\ i\neq j,\ r^{\sij}_{\rm d},\ r^{\sij}_{\rm u}>\epsilon\} \\
		%	&{\mathcal C}_{\rm OFDMA} \coloneqq \{\sijk : i=0,\ jk\neq0\in{\mathcal N},\ r^{\sij}_{\rm d},\ r^{\sij}_{\rm u}>\epsilon\}
		%\end{align}
		APはすべての組み合わせ$(i,j,k)$に対してスループット$r^{(i,j,k)}_{\rm d}$,$r^{(i,j,k)}_{\rm u1}$,$r^{(i,j,k)}_{\rm u2}$を推定する.
		ただし,$r^{(i,j,k)}_{\rm d}$はAPからSTA $i$への下り通信の推定スループット,$r^{(i,j,k)}_{\rm u1}$,$r^{(i,j,k)}_{\rm u2}$はSTA $j$,$k$による上り通信の推定スループットである.
		ここで,全組み合わせの集合から最低伝送速度の所要SINRを満たさない通信が含まれている組み合わせを除外した集合を$\mthc$とする,
		APは$\rijk=r^{(i,j,k)}_{\rm d}+r^{(i,j,k)}_{\rm u1}+r^{(i,j,k)}_{\rm u2}$とすべてのSTAの送信待機時間$d^{(j)}$を用いて,以下の最適化問題を解き各組み合わせで通信が行われる確率$p^{(i,j,k)}$を求める.
		ただし,送信待機時間とはあるデータフレームがバッファの先頭に到着してからの経過時間である.
		\begin{align}
			&{\mathcal P}_1: && {\rm max} \sum_{(i,j,k)\in{\mathcal C}} p^{(i,j,k)}r^{(i,j,k)}(d^{(j)}+d^{(k)})^{\alpha} &&&&&&\\
			&{\rm subject\ to} && \sum_{j,k\in\{j,k:(i,j,k)\in{\mathcal C}\}} p^{(i,j,k)} \geq \eta_{\rm d}^{(i)}, \forall i\in {\mathcal N}  \\
			&&& \sum_{i,j\in\{i,j:(i,j,k)\in{\mathcal C}\}} p^{(i,j,a)} +\sum_{i,k\in\{i,k:(i,j,k)\in{\mathcal C}\}} p^{(i,a,k)} \nonumber\\&&&\qquad- \sum_{i\in\{i:(i,j,k)\in{\mathcal C}\}} p^{(i,a,a)} \geq \eta_{\rm u}^{(a)}, \forall a\in {\mathcal N}  \\
			&&& \sum_{(i,j,k)\in{\mathcal C}} p^{(i,j,k)}=1 \\
			&{\rm variables:} &&p^{(i,j,k)} \in {\mathbb R}_{\geq 0},\forall(i,j,k)\in {\mathcal C}
		\end{align}
		\par
		目的関数は確率$\pijk$,推定スループット$\rijk$,STA $j$,$k$の送信待機時間の和$d^{(j)}+d^{(k)}$の積となっており,
		推定スループットが大きく,送信待機時間が大きなSTAを含む組ほど通信を行う確率が高くなるよう設計されている.
		また,$\alpha$は目的関数に対する送信待機時間の影響の大小を調節するための値であり,
		大きいほど送信待機時間の大きなSTAを含む組が選ばれやすくなる.
		一つ目の制約条件はあるSTA $i$が下り通信の送信先となる確率を$\eta_{\rm d}^{(i)}$以上とする条件であり,
		二つ目はあるSTA $a$が上り通信を行う確率を$\eta_{\rm u}^{(a)}$以上とする条件である.
		$\eta_{\rm d}$,$\eta_{\rm u}$は0より大きく,それぞれのトラヒックに比例した値が設定される.
		これらは,選ばれる確率が0となるSTAが発生しないようにするための条件である.
		APによって算出された確率$\pijk$はビーコンフレームによってSTAに通知される.
		\par
		次に,STA $i$,$j$,$k$を決定する方法を述べる.
		APは以下の式に従って,各STAが下り通信の送信先となる確率$p_{\rm d}^{(i)}$を求める.
		\begin{equation}
			p_{\rm d}^{(i)}= \sum_{j,k\in\{j,k:(i,j,k)\in{\mathcal C}\}}p^{(i,j,k)},\ \forall i \in \{0\}\cup{\mathcal N}
		\end{equation}
		APはこの確率$p_{\rm d}^{(i)}$に従って確率的にSTA $i$を選択する.
		ここで下り通信の送信先として決定したSTAをSTA $i^*$とする.
		このとき$i^*=0$であれば,下り通信が行われないことを示す.
		STA $i^*$の決定後APはSTA $i^*$へ送信するデータフレームのヘッダ部分のみを送信し,送信先がSTA $i^*$であることを全STAに通知する.
		STA $i^*$以外のSTAは以下の条件付き確率を計算する.
		\begin{align}
			p_{\rm u1}^{(i^*,j,k)}=\left(\sum_{k\in\{k:(i^*,j,k)\in\mthc\}} p^{(i^*,j,k)}\right) / p_{\rm d}^{(i)}, \\
			\qquad\qquad\forall j \in \{0\}\cup{\mathcal N}\backslash \{i^*\}
		\end{align}
		これは,APがSTA $i^*$へ送信することが決まった上で,各STA $j$が上り通信を行う確率である.
		この条件付き確率をもとに,各STAはコンテンションウィンドウサイズ${\rm CW}^{\sijk}_{\rm u1}$を
		\begin{equation}
			{\rm CW}^{(i^*,j,k)}_{\rm u1} = \lceil 1/p_{\rm u1}^{(i^*,j,k)} \rceil
		\end{equation}
		と設定する.
		ただし,$\lceil x \rceil$は$x$を超えない最大の整数である.
		各STAは$[0,\ {\rm CW}^{(i^*,j,k)}_{\rm u1}]$の一様分布から生成されるバックオフカウンタ$w_{\rm u1}^{(i^*,j,k)}$を設定し,
		CSMA/CAのバックオフアルゴリズムを用いてバックオフカウンタを1ずつ減らす.
		その結果,最初にカウンタが0となったSTAが上り通信を行う.
		このSTAをSTA $j^*$とすると,STA $j^*$はAPへ送信するデータフレームのヘッダ部分のみを送信し,
		他のSTAに自身が上り通信を行うことを通知する.
		残りのSTAは,STA $j$の決定の際と同様に以下の条件付き確率を求める.
		\begin{align}
			p_{\rm u2}^{(i^*,j^*,k)}=p^{(i^*,j^*,k)} / \left(\sum_{k\in\{k:(i^*,j^*,k)\in\mthc\}} p^{(i^*,j^*,k)}\right)\\
			\qquad\qquad\forall k \in \{0\}\cup{\mathcal N}\backslash \{i^*,j^*\}
		\end{align}
		これは,STA $i^*$,STA $j^*$が通信に参加することが決まった上での各STAがSTA $k$として上り通信を行う確率である.
		以降STA $j$を決定する際と同様に${\rm CW}^{(i^*,j^*,k)}_{\rm u2}$,$w_{\rm u2}^{(i^*,j^*,k)}$を設定し,
		最小の$w_{\rm u2}^{(i^*,j^*,k)}$を設定したSTAがSTA $k^*$となり,組み合わせが決まる.
		組み合わせ決定後,AP,STA $j^*$,$k^*$はそれぞれデータフレームを送信する.

	\subsection{計算時間の削減}\label{sec:time}
		\begin{figure}[t]
			\centering
			\epsfig{file=fig/time.eps, scale=0.6}
			\caption{システムスループット}
			\label{fig:time_image}
		\end{figure}
		本節では,最適化問題を解くための計算時間を削減する手法について検討する.
		APは前節の最適化問題を定期的に解いて,確率$\pijk$を更新しなければならない.
		なぜなら,目的関数に含まれる送信待機時間$d^{(j)}$,$d^{(k)}$はSTAが送信を行う度に変化する値であるためである.
		計算結果はビーコンフレームによって全STAに通知されるが,一般的なビーコン周期は100\,msであり,
		ビーコン毎に最適化問題を解く場合は最大で100\,ms以内に解かねばならず,
		仮にビーコン数回毎に更新するように計算頻度を減らしたとしても数百ミリ秒単位で解けなければならない.
		そのため,計算時間を削減することを考えなければならない.
		\par
		一般的に最適化問題の計算時間は変数の数に依存する.
		提案方式における変数は$\pijk$であり,この数は組み合わせの数$|\mthc|$と同じである.
		したがって,集合$\mthc$に含まれる組み合わせの数を減らすことができれば計算時間を削減することができる.
		最適化の結果に与える影響をできるだけ小さくしながら組み合わせの数を減らすためには,
		ユーザ間干渉が大きく,例え通信ができたとしても伝送速度が低くなってしまうような組み合わせを除けばよい.
		\par
		そこで,本稿では計算時間の削減を行うためにSTAを位置によってグループ分けを行い,その中から組み合わせを決定するという方法をとる.
		具体的には,図\ref{fig:time_image}のようにAPを中心とした直行座標を設定し,それぞれのSTAがどの象限に位置するかによって4つのグループに分ける.
		そして,組み合わせの集合$\mthc$には,OFDMAとUFD通信を組み合わせる場合はSTA $i$と2台のSTA $j$,$k$は対角の象限,STA $j$と$k$は同じ象限に存在するような組み合わせ,UFD通信の場合はSTA$i$と$j$は対角の象限に存在する組み合わせのみ含める.
		ユーザ間干渉はSTA $i$とSTA $j$,$k$の間の距離が小さいほど大きくなるため,
		STA $i$とSTA$j$,$k$が隣り合った象限に存在する組み合わせはユーザ間干渉が大きくなる可能性が高い.
		そのため,そういった組み合わせは選択される可能性が低く,選択対象から除外しても影響は小さいと考えられる.
		\par
		ただし,本稿においてはSTAの位置は既知であるとし,時間変化もないものとする.


\section{シミュレーション評価}

	\begin{table}[t]
		\centering
		\caption{シミュレーション諸元}
		\label{tab:param}
		\begin{tabular}{cc} \hline
			領域の大きさ $L$ & 100\,m \\
			伝送速度 & シャノン容量 \\
			送信電力 & 15\,dBm \\
			雑音指数 & 10\,dB \\
			周波数帯 $f$& 2.4\,GHz \\
			帯域幅 $B$ & 20\,MHz \\
			伝搬損失 & $30\log D + 40$\\
			&($D$: 送受信点間距離)\\
			自己干渉除去 & 110\,dB \\
			シミュレーション時間 & 10\,s \\\hline
		\end{tabular}
	\end{table}

	本章では提案手法の有効性をシミュレーションによって評価する.
	半二重通信のみを用いる場合,従来方式~\cite{promac_fair}を用いて半二重通信とUFD通信を併用する場合と,
	提案方式を比較する.
	OFDMAによる多重化は二多重までとし,チャネル幅は二等分するものとする.
	図\ref{fig:model}のように,1台のAPが $L=$100\,m四方の領域の中心に設置され,その周りに$N$台のSTAがランダムに配置されているとする.
	第\ref{sec:opt}節で述べた部分以外のMACプロトコルは~\cite{promac}に従うものとする.
	伝送速度はIEEE 802.11aに従う.
	上下通信ともに飽和トラヒックであり,APには1500\,Bの,STAには64\,Bのデータフレームが発生しているものとする.
	また,本稿では遅延時間の代わりに送信待機時間を評価指標とする.

	\subsection{遅延時間の削減}
		\begin{figure}[t]
			\centering
			\epsfig{file=graph/delay.eps, scale=0.6}
			\caption{STAの平均送信待機時間}
			\label{fig:delay}
		\end{figure}
		\begin{figure}[t]
			\centering
			\epsfig{file=graph/thr.eps, scale=0.6}
			\caption{システムスループット}
			\label{fig:thr}
		\end{figure}

		$N=50$台の場合について,図\ref{fig:delay},\ref{fig:thr}にSTAの平均送信待機時間とシステムスループットを示す.
		第\ref{sec:problem}章で述べた通り,従来方式の送信待機時間は半二重通信の場合のそれと比べ大きいことがわかる.
		一方,提案方式はパラメータ$\alpha$を大きくすることで送信待機時間を半二重通信と同等の値まで削減できていることがわかる.
		しかし,$\alpha$が大きくなるにつれて,システムスループットが低下してしまっている.
		これは,送信待機時間$d^{(j)},\ d^{(k)}$の項の影響が大きくなり,
		送信待機時間が大きいSTAに通信させるためにOFDMA単体による通信が多くなることが原因であると考える.

	\subsection{計算時間の削減}
		\begin{table}[t]
			\centering
			\caption{最適化問題を1回解くために必要な平均時間}
			\label{tab:time}
			\begin{tabular}{cc}
			 計算時間削減なし & 計算時間削減あり\\ \hline
			 803\,ms & 483\,ms \\\hline
			\end{tabular}
		\end{table}
		\begin{figure}[t]
			\centering
			\epsfig{file=graph/thr_time.eps, scale=0.6}
			\caption{システムスループットへの計算時間削減手法の影響}
			\label{fig:thr_time}
		\end{figure}

	次に計算時間について評価する.
	表\ref{tab:time}に第\ref{sec:time}節で述べた計算時間削減手法を用いた場合と用いない場合について,
	最適化問題を一回解くために必要な平均時間を示す.
	結果から分かる通り,計算時間を40\%削減できている.
	さらに,計算時間削減手法を用いたことによる影響を調べるために,図\ref{fig:thr_time}に両者のシステムスループットを示す.
	全体的に計算時間削減手法を用いた方が用いない場合に比べてシステムスループットが小さくなっている.
	差は$\alpha=0$のときには1\%,$\alpha=1$のときには18\%となっている.
	$\alpha$が小さいときには送信待機時間$d^{(j)},\ d^{(k)}$の影響が小さく,
	干渉の小さな組み合わせが選ばれやすい.
	干渉の小さな組み合わせは,APに対してSTA $i$とSTA $j$,$k$が対角の位置に存在している場合であり,
	このような組み合わせは計算時間削減手法を用いていても除外されていないため,影響が少ないと考えられる.
	一方,$\alpha$が大きいときには送信待機時間$d^{(j)},\ d^{(k)}$の影響が大きく,
	例えある程度干渉が大きい組み合わせでも送信待機時間が大きいSTAを選ばざる負えない.
	このような場合は,計算時間削減手法によって除外された組み合わせの中によりよい組み合わせが含まれてしまっていることがあるため,
	システムスループットが低下してしまうと考えられる.

\section{まとめ}
	本稿では,OFDMAとUFD通信を組み合わせて用いる手法を提案した.
	STAの遅延時間が増大するというUFD通信の問題点を,上り通信をOFDMAにより多重化し,
	STAの送信機会を増加させることで遅延時間の削減を行った.
	更に,STAのグループ分けを行い最適化問題の変数の数を減らすことで,計算時間を削減する手法についても述べた.

\bibliographystyle{sieicej}
\bibliography{main2}

\end{document}
